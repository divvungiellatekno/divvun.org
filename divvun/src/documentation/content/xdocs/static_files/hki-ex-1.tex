%%%%%%%%%%%%%%%%%%%%%%%%%%%%%%%%%%%%%%%%%%%%%%%%%%%%%%%%%%%%%%
% $Id$ %
%%%%%%%%%%%%%%%%%%%%%%%%%%%%%%%%%%%%%%%%%%%%%%%%%%%%%%%%%%%%%%
\documentclass[a4paper,finnish]{article}
\usepackage{man}
\begin{document}

\title{Sopimus tekstien käyttöoikeuden luovuttamisesta} 

\author{}
\date{}
\maketitle

\section{SOPIMUSOSAPUOLET}

\subsection{Käyttöoikeuden luovuttajat}

\begin{quote}
  \def\arraystretch{1.2}
  \begin{tabular}{lcl}
    &&\hbox to 20em{}\\
    Rooli&:& (tekijä/kääntäjä/kustantaja)\\
    Nimi&:&\\
    \cline{3-3}
    Jakeluosoite&:&\\
    \cline{3-3}
    Postitoimipaikka&:&\\
    \cline{3-3}
  \end{tabular}
\end{quote}

\begin{quote}
  \def\arraystretch{1.2}
  \begin{tabular}{lcl}
    &&\hbox to 20em{}\\
    Rooli&:& (tekijä/kääntäjä/kustantaja)\\
    Nimi&:&\\
    \cline{3-3}
    Jakeluosoite&:&\\
    \cline{3-3}
    Postitoimipaikka&:&\\
    \cline{3-3}
  \end{tabular}
\end{quote}

\subsection{Käyttöoikeudensaaja}
\begin{quote}
  \def\arraystretch{1.2}
\begin{tabular}{lcl}
    &&\hbox to 20em{}\\
    Nimi&:&\\
    \cline{3-3}
    Jakeluosoite&:&\\
    \cline{3-3}
    Postitoimipaikka&:&\\
    \cline{3-3}
\end{tabular}
\end{quote}

\section{SOPIMUKSEN TARKOITUS}

\paragraph{Tämä sopimus koskee sähköisten tai sähköiseen muotoon saatettavien
\termi{tekstien} käyttöoikeuden luovuttamista siinä laajuudessa, että tekstit
voidaan liittää \termi{käyttöoikeudensaajan} ylläpitämään
\termi{tekstikokoelmaan} ja niitä voidaan käyttää \termi{tekstikokoelman} osana
tutkimuksellisiin tarkoituksiin tämän sopimuksen ehtojen mukaisesti.}

\paragraph{\termi{Käyttöoikeuden luovuttajat} vakuuttavat, että heillä on
oikeus myöntää \termi{käyttöoikeudensaajalle} tämän sopimuksen
kohteena olevien \termi{tekstien} edellisessä kappaleessa määritellyn
laajuinen käyttöoikeus.}

\section{KÄYTTÖOIKEUDEN LAAJUUS}

\paragraph{\termi{Tekstit}, joiden käyttöoikeus luovutetaan tällä sopimuksella 
on yksilöity ja kuvattu tämän sopimuksen liitteessä 1.}

\paragraph{Sopimuksen kohteena olevien \termi{tekstien} tekijän- ja
immateriaalioikeudet kuuluvat luovuttajalle.}

\paragraph{\termi{käyttöoikeudensaajalle} myönnetään vapaa oikeus
veloituksetta liittää \termi{tekstit} \termi{tekstikokoelmaan}. Tätä
varten tekstejä voidaan käsitellä sekä käsin että koneellisesti, mutta
siten ettei niiden sisältöä muuteta. \termi{Teksteihin} voidaan
kuitenkin liittää tieteellisessä tutkimuksessa tai atk"-käsittelyssä
tarvittavaa \termi{koodausta} eli tietoja esimerkiksi virkkeistä,
kappaleista, sanaluokista ym. kielellisistä ominaisuuksista tai muista
luokituksista.}

\paragraph{\termi{käyttöoikeuden luovuttajalle} ei muodostu tekijän- tai muita
immateriaalioikeuksia \termi{tekstien} käsittelyn avulla saavutettuihin
tutkimustuloksiin. \termi{tekstikokoelman} rakentamista edistäneet tahot
kuitenkin kuvataan \termi{tekstikokoelman} esitteissä ja dokumentaatiossa.}

\paragraph{\termi{Käyttöoikeudensaaja} voi tallettaa \termi{tekstit}
omassa hallinnassaan olevalle tietokoneelle ja/tai sijoittaa \termi{tekstit}
tietojenkäsittelystä vastaavalle ulkopuoliselle atk-palvelulaitokselle
siten, että \termi{käyttöoikeudensaaja} sitoutuu varmistamaan, että
\termi{tekstien} tietoturva sekä tämän sopimuksen ehdot ja
velvoitukset täyttyvät.} 

\paragraph{\termi{Käyttöoikeudensaajalla} ei ole oikeutta luovuttaa
\termi{tekstejä}, niiden kopioita tai niiden käyttöoikeutta edelleen
kolmansille osapuolille, muuten kuin pykälässä 3.7 mainitussa tarkoituksessa.}

\paragraph{\termi{Käyttöoikeudensaaja} voi myöntää \termi{tekstikokoelman}
henkilökohtaisen käyttöluvan henkilölle, joka on allekirjoittanut
liitteenä 2 olevan käyttöoikeussopimuksen. \termi{Käyttöoikeudensaaja}
ei myönnä \termi{tekstikokoelman} käyttölupaa henkilölle, jonka on perusteltua
epäillä laiminlyövän sopimuksen velvotteita. \termi{Käyttöoikeudensaaja}
sitoutuu mahdollisen rikkomuksen tapahtuessa informoimaan ensi tilassa
\termi{käyttöoikeuden luovuttajaa}}

\paragraph{\termi{Käyttöoikeudensaaja} sitoutuu omalta osaltaan noudattamaan
sellaista hyvää tietojenkäsittelytapaa, jolla sovitulla tavalla
huolehditaan tietoturvallisuudesta ja tietosuojasta.}

\paragraph{\termi{Käyttöoikeudensaaja} tarjoutuu toimittamaan
\termi{tekstien} muokatun ja mahdollisilla koodeilla varustetun
\termi{tekstikokoelmaan} liitetyn version kopion \termi{käyttöoikeuden
luovuttajalle}.}

\paragraph{Tämä sopimus astuu voimaan, kun kaikki sopimusosapuolet ovat sen
allekirjoittaneet.}

\paragraph{\termi{käyttöoikeudensaaja} voi purkaa tämän sopimuksen
palauttamalla \termi{tekstit} kopioineen \termi{käyttöoikeuden
luovuttajalle} tai hävittämällä \termi{tekstit} kopioineen ja
ilmoittamalla tästä asianmukaisesti \termi{käyttöoikeuden
luovuttajalle}. \termi{Käyttöoikeuden luovuttajalla} on oikeus purkaa
tämä sopimus, jos \termi{käyttöoikeudensaaja} olennaisesti rikkoo tämän
sopimuksen velvoituksia vastaan, eikä korjaa rikkomustaan
kirjallisesta huomautuksesta huolimatta 60 päivän kuluessa
huomautuksen saamisesta.}

\paragraph{Sopimussuhteesta aiheutuvat erimielisyydet pyritään
ratkaisemaan ensi sijassa neuvotteluin.}

\paragraph{Tästä sopimuksesta on tehty kaksi samasanaista kappaletta,
yksi kummallekin sopimusosapuolelle.}

\section{ALLEKIRJOITUKSET}

\begin{tabular}{ll}
Käyttöoikeuden luovuttaja & Käyttöoikeudensaaja \\
 & \\
Paikka ja aika: \rule{2in}{.01in} & Paikka ja aika: \rule{2in}{.01in} \\ \\
Allekirjoitus : \rule{2in}{.01in} & Allekirjoitus : \rule{2in}{.01in} \\ \\
\end{tabular}


\section{LIITTEET}
Liite 1. Tekstien kuvaukset \\
Liite 2. Tekstikokoelman käyttöoikeussopimus \\
\end{document}
